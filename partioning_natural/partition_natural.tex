\documentclass[12pt]{article}
\usepackage{amsmath, amssymb}

\begin{document}

\textbf{Partitioning Natural Numbers into Equal-Sum Subsets}

Let $n \in \mathbb{N}$ be a positive integer, and let $M \in \mathbb{N}$ with $M < n$.

We ask: under what conditions can the set
\[
S_n = \{1, 2, 3, \dots, n\}
\]
be partitioned into $M$ disjoint subsets, each having the same sum?

\medskip

The total sum of the elements of $S_n$ is
\[
1 + 2 + \cdots + n \;=\; \frac{n(n+1)}{2}.
\]

For a partition into $M$ equal-sum subsets to exist, each subset must sum to
\[
\frac{n(n+1)}{2M}.
\]

\medskip

\noindent \textbf{Necessary condition:}  
\[
\frac{n(n+1)}{2M} \in \mathbb{Z}.
\]

Equivalently,
\[
2M \;\bigm|\; n(n+1).
\]

Thus a partition into $M$ equal-sum subsets is possible only when $2M$ divides either $n$ or $(n+1)$.

\end{document}
